\renewcommand{\theequation}{\theenumi}
\begin{enumerate}[label=\thesubsection.\arabic*.,ref=\thesubsection.\theenumi]
\numberwithin{equation}{enumi}

\begin{figure}[!ht]
\centering
\resizebox{\columnwidth}{!}{\input{./figs/triangle_ex/triangle_fig}}
\caption{Triangle by Latex-Tikz}
\label{fig:triangle_triangle_ex}	
\end{figure}

\item The figure obtained looks like Fig. \ref{fig:triangle_triangle_ex}.\\ 

\item The design parameters used for construction See Table. \ref{table:table1_triangle_ex}.
\begin{table}[ht!]
\centering
\input{./tables/triangle_ex/design_params}
\caption{Triangle $ABC$}
\label{table:table1_triangle_ex}	
\end{table} 

\item To find the coordinates of $\vec{C}$ in Fig \ref{fig:triangle_triangle_ex}\\
\solution
\begin{align}
\vec{A} &=\myvec{0\\0} \label{eq:constr_a_triangle_ex}\\
\vec{B} &=\myvec{b\\0} \label{eq:constr_b_triangle_ex}\\
\vec{C} &=\myvec{x\\y} \label{eq:constr_c_triangle_ex}
%\vec{B} - \vec{A} &= \myvec{a\\0}\\
%\vec{C} - \vec{A} &= \myvec{x\\y}\\
%\vec{C} - \vec{B} &= \myvec{x - a\\y}\\
%\vec{A} - \vec{B} &= \myvec{-a\\0} 
\end{align}

Finding the Scalar Products:
\begin{align}
\begin{split}\label{eq:scalar1_triangle_ex}
\brak{\vec{B} - \vec{A}}^T\brak{\vec{C}-\vec{A}} &= \\ \|\vec{B} - \vec{A}\|
\ \| \vec{C}-\vec{A} \| \cos{\theta}
\end{split}\\ \nonumber\\
\begin{split}\label{eq:scalar2_triangle_ex}
\brak{\vec{C} - \vec{B}}^T\brak{\vec{A}-\vec{B}} &= \\ 
\|\vec{C} - \vec{B}\|\ \| \vec{A}-\vec{B} \| \cos{\alpha}
\end{split}
\end{align}

On simplifying equation \ref{eq:scalar1_triangle_ex} and \ref{eq:scalar2_triangle_ex}:
\begin{align}
x^2 \tan{\theta}^2 &= y^2 \label{eq:scalar3_triangle_ex}\\
\brak{x - a}^2 &= \brak{\brak{x-a}^2 + y^2}\cos{\alpha}^2\label{eq:scalar4_triangle_ex}
\end{align}
Substituting \ref{eq:scalar3_triangle_ex} in \ref{eq:scalar4_triangle_ex}:
\begin{align}
\begin{split}\label{eq:scalar5_triangle_ex}
x^2\brak{1-\cos{\alpha}^2-\tan{\theta}^2\cos{\alpha}^2}\\
+ x\brak{2a\cos{\alpha}^2-2a} + a^2\sin{\alpha}^2
\end{split}
\end{align}\\

If $\theta$ and $\alpha$ are accute angles:
\begin{align}
x &= \frac{\brak{-b-\sqrt{b^2-4ac}}}{2a}
\end{align}
else:
\begin{align}
x &= \frac{\brak{-b+\sqrt{b^2-4ac}}}{2a}
\end{align}\\

The value of $x$ can then be substituted in \ref{eq:scalar3_triangle_ex} to find the coordinates of $\vec{C}$\\ \\

\item
From the given information, 
The values are listed in \ref{table:table2_triangle_ex}\\
\begin{table}[ht!]
\centering
\input{./tables/triangle_ex/result}
\caption{Value of $\vec{C}$}
\label{table:table2_triangle_ex}	
\end{table} 

\item Draw Fig. \ref{fig:triangle2_triangle_ex}.

\begin{figure}[!ht]
\centering
\includegraphics[width=\columnwidth]{./figs/triangle_ex/triangle_linearalg.eps}
\caption{Triangle generated using python}
\label{fig:triangle2_triangle_ex}
\end{figure} 

\solution The  following Python code generates Fig. \ref{fig:triangle2_triangle_ex}

\begin{lstlisting}
codes/triangle_ex/triangle_linearalg.py
\end{lstlisting}

and the equivalent latex-tikz code generating Fig. \ref{fig:triangle2_triangle_ex} is 
\begin{lstlisting}
figs/triangle_ex/triangle_fig.tex
\end{lstlisting}
%
The above latex code can be compiled as a standalone document as
\begin{lstlisting}
figs/triangle_ex/triangle_final.tex
\end{lstlisting}
\end{enumerate}


\renewcommand{\theequation}{\theenumi}
\begin{enumerate}[label=\thesubsection.\arabic*.,ref=\thesubsection.\theenumi]
\numberwithin{equation}{enumi}

\begin{figure}[!ht]
\centering
\resizebox{\columnwidth}{!}{\input{./figs/circle_ex/circumcircle_fig}}
\caption{Circumcircle by Latex-Tikz}
\label{fig:Circumcircle_circle_ex}	
\end{figure}

\item The figure obtained looks like Fig. \ref{fig:Circumcircle_circle_ex}.\\ 

\item Coordinates of $\triangle ABC$ Fig\ref{fig:Circumcircle_circle_ex}. \\
\begin{align}
\vec{A} &=\myvec{6\\-6} \label{eq:constr_a_circle_ex}\\
\vec{B} &=\myvec{3\\-7} \label{eq:constr_b_circle_ex}\\
\vec{C} &=\myvec{3\\3} \label{eq:constr_c_circle_ex}
\end{align}

\item To find the coordinates of $\vec{O}$. \\
\solution A circle passing through three non-collinear points is the circumcircle and the center is the circumcenter.\\
\begin{align}
\vec{O} &= \frac{A\sin{\angle 2A} + B\sin{\angle 2B} + C\sin{\angle 2C}}{\sin{\angle 2A} + \sin{\angle 2B} + \sin{\angle 2C}}\\
\cos{\angle A} &= \frac{ \brak{\vec{C}-\vec{A}}^T \brak{\vec{B}-\vec{A}}}{\| \vec{C}-\vec{A} \|\ \| \vec{B}-\vec{A} \|} \\
\cos{\angle B} &= \frac{ \brak{\vec{A}-\vec{B}}^T \brak{\vec{C}-\vec{B}}}{\| \vec{A}-\vec{B} \|\ \| \vec{C}-\vec{B} \|} \\
\cos{\angle C} &= \frac{ \brak{\vec{A}-\vec{C}}^T \brak{\vec{B}-\vec{C}}}{\| \vec{A}-\vec{C} \|\ \| \vec{B}-\vec{C} \|} 
\end{align}

\item
From the given information, 
The values are listed in \ref{table:table2_circle_ex}\\
\begin{table}[ht!]
\centering
\input{./tables/circle_ex/result}
\caption{Value of $\vec{O}$}
\label{table:table2_circle_ex}	
\end{table} 

%\pagebreak
\item Draw Fig. \ref{fig:Circumcircle2_circle_ex}.

\begin{figure}[!ht]
\centering
\includegraphics[width=\columnwidth]{./figs/circle_ex/circumcircle.eps}
\caption{Circumcircle generated using python}
\label{fig:Circumcircle2_circle_ex}
\end{figure} 

\solution The  following Python code generates Fig. \ref{fig:Circumcircle2_circle_ex}

\begin{lstlisting}
codes/circle_ex/circumcircle.py
\end{lstlisting}

and the equivalent latex-tikz code generating Fig. \ref{fig:Circumcircle2_circle_ex} is 
\begin{lstlisting}
figs/circle_ex/circumcircle_fig.tex
\end{lstlisting}
%
The above latex code can be compiled as a standalone document as
\begin{lstlisting}
figs/circle_ex/circumcircle_fig_final.tex
\end{lstlisting}
\end{enumerate}


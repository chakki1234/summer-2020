
\renewcommand{\theequation}{\theenumi}
\begin{enumerate}[label=\thesection.\arabic*.,ref=\thesection.\theenumi]
\numberwithin{equation}{enumi}

\item \solution 
\begin{flushleft}
The given equation can be represented as follows in the vector form:
\end{flushleft}
\begin{align}
\begin{pmatrix}
5 & -1 
\end{pmatrix}
x + 5 = 0
\end{align}

To find the roots $y=0$:
\begin{align}
x + 5 &= 0 \\
x &= -5
\end{align}

\item \solution 
\begin{flushleft}
The given equation can be represented as follows in the vector form:
\end{flushleft}
\begin{align}
\begin{pmatrix}
5 & -1 
\end{pmatrix}
x - 5 = 0
\end{align}

To find the roots $y=0$:
\begin{align}
x - 5 &= 0 \\
x &= 5
\end{align}

\item \solution 
\begin{flushleft}
The given equation can be represented as follows in the vector form:
\end{flushleft}
\begin{align}
\begin{pmatrix}
2 & -1 
\end{pmatrix}
x + 5 = 0
\end{align}

To find the roots $y=0$:
\begin{align}
2x + 5 &= 0 \\
x &= \frac{-5}{2}
\end{align}

\item \solution 
\begin{flushleft}
The given equation can be represented as follows in the vector form:
\end{flushleft}
\begin{align}
\begin{pmatrix}
3 & -1 
\end{pmatrix}
x - 2 = 0
\end{align}

To find the roots $y=0$:
\begin{align}
3x - 2 &= 0 \\
x &= \frac{2}{3}
\end{align}

\item \solution 
\begin{flushleft}
The given equation can be represented as follows in the vector form:
\end{flushleft}
\begin{align}
\begin{pmatrix}
3 & -1 
\end{pmatrix}
x  = 0
\end{align}

To find the roots $y=0$:
\begin{align}
3x  &= 0 \\
x &= 0
\end{align}

\end{enumerate}

%\solution 
%\begin{align}
%\vec{A} &=\myvec{0\\0} \label{eq:constr_a}\\
%\vec{B} &=\myvec{36\\15} \label{eq:constr_b}
%\end{align}
%Distance between $\vec{A}$ and $\vec{B}$ is:
%\begin{align}
%\| \vec{A} - \vec{B} \| 
%\end{align}
%From the given information:
%\begin{align}
%\| \myvec{0\\0} - \myvec{36\\15} \| = 39
%\end{align}







\renewcommand{\theequation}{\theenumi}
\begin{enumerate}
\numberwithin{equation}{enumi}

%\begin{figure}[!ht]
%\centering
%\resizebox{\columnwidth}{!}{\input{./figs/line_ex/pts_and_vectors/dist_bt_pts.eps_fig}}
%\caption{$AB$ by Latex-Tikz}
%\label{fig:dist_btw_pts_pts_and_vectors}	
%\end{figure}

\item The figure obtained looks like Fig. \ref{fig:dist_btw_pts2_pts_and_vectors}.\\ 

\item The coordinates are: \\
\begin{align}
\vec{A} &=\myvec{0\\0} \label{eq:constr_a_pts_and_vectors}\\
\vec{B} &=\myvec{36\\15} \label{eq:constr_b_pts_and_vectors}
\end{align}

\item Draw Fig. \ref{fig:dist_btw_pts2_pts_and_vectors}.

\begin{figure}[!ht]
\centering
\includegraphics[width=\columnwidth]{./figs/line_ex/pts_and_vectors/dist_bt_pts.eps}
\caption{$AB$ generated using python}
\label{fig:dist_btw_pts2_pts_and_vectors}
\end{figure} 

\solution The  following Python code generates Fig. \ref{fig:dist_btw_pts2_pts_and_vectors}

\begin{lstlisting}
codes/line_ex/pts_and_vectors/dist_btw_pts.py
\end{lstlisting}

and the equivalent latex-tikz code generating Fig. \ref{fig:dist_btw_pts2_pts_and_vectors} is 
\begin{lstlisting}
figs/line_ex/pts_and_vectors/dist_bt_pts.eps_fig.tex
\end{lstlisting}
%
The above latex code can be compiled as a standalone document as
\begin{lstlisting}
figs/line_ex/pts_and_vectors/dist_bt_pts.eps_fig_final.tex
\end{lstlisting}
\end{enumerate}


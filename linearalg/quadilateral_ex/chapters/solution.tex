\begin{theorem}\label{theorem:th1}
Sum of opposite angles in a cyclic quadilateral equals $180\degree$.
\end{theorem}
\solution  From theorem \ref{theorem:th1}
\begin{align}
\angle A + \angle C &=  180\degree\\
\angle B + \angle D &=  180\degree
\end{align}

From the given information:
\begin{align}
4y+20-4x &= 180\degree \label{eq:sol1} \\
3y-5-7x+5 &= 180\degree \label{eq:sol2}
\end{align}

Solving equations \ref{eq:sol1} and \ref{eq:sol2}:
\begin{align}
x &= -15 \\
y &= 25 \\
\implies \angle A &= 120\degree \\
\implies \angle B &= 70\degree \\
\implies \angle C &= 60\degree \\
\implies \angle D &= 110\degree 
\end{align}





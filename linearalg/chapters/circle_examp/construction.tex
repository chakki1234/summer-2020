\renewcommand{\theequation}{\theenumi}
\begin{enumerate}[label=\thesubsection.\arabic*.,ref=\thesubsection.\theenumi]
\numberwithin{equation}{enumi}

\begin{figure}[!ht]
\centering
\resizebox{\columnwidth}{!}{\input{./figs/circle_examp/circle_fig}}
\caption{Circle by Latex-Tikz}
\label{fig:circle_circle_examp}	
\end{figure}

\item The figure obtained looks like Fig. \ref{fig:circle_circle_examp}.\\ 

\item The general of a circle equation is $Ax^2 + Bxy + Ay^2 + Dx + Ey + F$, the equation can be represented as follow in the vector form:
\begin{align}
x^T 
\begin{pmatrix}
A & \frac{B}{2} \\
\frac{B}{2} & A
\end{pmatrix}
x + 
\begin{pmatrix}
D & E 
\end{pmatrix}
x + F = 0
\end{align}
To find the center - $\vec{O}$ and radius - $r$ of a circle:
\begin{align}
\vec{O} &= \frac{-1}{2A}\begin{pmatrix}
D & E 
\end{pmatrix}\label{eq:circle_example}\\
r &= \frac{1}{A}\sqrt{ \frac{1}{4} \| \myvec{D\\ E}\ \|^2 - F^2}\label{eq:circle_example2}
\end{align}

\item
From the given information, 
The values are listed in \ref{table:table_circle_examp}\\
\begin{table}[ht!]
\centering
\input{./tables/circle_examp/result}
\caption{Value of $\vec{O}$ and $r$}
\label{table:table_circle_examp}	
\end{table} 

\item Draw Fig. \ref{fig:circle2_circle_examp}.

\begin{figure}[!ht]
\centering
\includegraphics[width=\columnwidth]{./figs/circle_examp/circle.eps}
\caption{Circle generated using python}
\label{fig:circle2_circle_examp}
\end{figure} 

\solution The  following Python code generates Fig. \ref{fig:circle2_circle_examp}

\begin{lstlisting}
codes/circle_exam.py
\end{lstlisting}

and the equivalent latex-tikz code generating Fig. \ref{fig:circle2_circle_examp} is 
\begin{lstlisting}
figs/circle_fig.tex
\end{lstlisting}
%
The above latex code can be compiled as a standalone document as
\begin{lstlisting}
figs/circle_fig_final.tex
\end{lstlisting}
\end{enumerate}


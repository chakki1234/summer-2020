\renewcommand{\theequation}{\theenumi}
\begin{enumerate}[label=\thesubsection.\arabic*.,ref=\thesubsection.\theenumi]
\numberwithin{equation}{enumi}

\begin{figure}[!ht]
\centering
\resizebox{\columnwidth}{!}{\input{./figs/quadilateral_ex/cyclic_quad_fig}}
\caption{Cyclic quadilateral by Latex-Tikz}
\label{fig:cyclic_quad_quadilateral_ex}	
\end{figure}

\item The figure obtained looks like Fig. \ref{fig:cyclic_quad_quadilateral_ex}.\\ 

\item The design parameters used for construction See Table. \ref{table:table1_quadilateral_ex}.
\begin{table}[ht!]
\centering
\input{./tables/quadilateral_ex/design_params}
\caption{Quadilateral $ABCD$}
\label{table:table1_quadilateral_ex}	
\end{table} 

\item Coordinates of cyclic quadilateral Fig\ref{fig:cyclic_quad_quadilateral_ex}. \\
\begin{align}
\vec{A} &=\myvec{0\\0} \label{eq:constr_a_quadilateral_ex}\\
\vec{B} &=\myvec{a\\0} \label{eq:constr_b_quadilateral_ex}\\
\vec{C} &=\myvec{x\\y} \label{eq:constr_c_quadilateral_ex}\\
\vec{D} &= \myvec{b\cos{\theta}\\b\sin{\theta}}\label{eq:constr_d_quadilateral_ex}
\end{align}

\item To find the coordinates of $\vec{C}$. 
\begin{theorem}
Angles formed in the same segment of a circle are always equal in measure.
\end{theorem} 
\begin{align}
\cos{\gamma} &= \frac{\brak{A-D}^T\brak{B-D}}{\| A -D \| \ \|B-D\|} \\
\theta &= 180\degree - \gamma - \angle B
\end{align}

In $\triangle ACB$. Finding the Scalar Products:
\begin{align}
\begin{split}\label{eq:scalar1_quadilateral_ex}
\brak{\vec{B} - \vec{A}}^T\brak{\vec{C}-\vec{A}} &= \\ \|\vec{B} - \vec{A}\|
\ \| \vec{C}-\vec{A} \| \cos{\theta}
\end{split}\\ \nonumber\\
\begin{split}\label{eq:scalar2_quadilateral_ex}
\brak{\vec{C} - \vec{B}}^T\brak{\vec{A}-\vec{B}} &= \\ 
\|\vec{C} - \vec{B}\|\ \| \vec{A}-\vec{B} \| \cos{\alpha}
\end{split}
\end{align}

On simplifying equation \ref{eq:scalar1_quadilateral_ex} and \ref{eq:scalar2_quadilateral_ex}:
\begin{align}
x^2 \tan{\theta}^2 &= y^2 \label{eq:scalar3_quadilateral_ex}\\
\brak{x - a}^2 &= \brak{\brak{x-a}^2 + y^2}\cos{\alpha}^2\label{eq:scalar4_quadilateral_ex}
\end{align}
Substituting \ref{eq:scalar3_quadilateral_ex} in \ref{eq:scalar4_quadilateral_ex}:
\begin{align}
\begin{split}\label{eq:scalar5_quadilateral_ex}
x^2\brak{1-\cos{\alpha}^2-\tan{\theta}^2\cos{\alpha}^2}\\
+ x\brak{2a\cos{\alpha}^2-2a} + a^2\sin{\alpha}^2
\end{split}
\end{align}\\

If $\theta$ and $\alpha$ are accute angles:
\begin{align}
x &= \frac{\brak{-b-\sqrt{b^2-4ac}}}{2a}
\end{align}
else:
\begin{align}
x &= \frac{\brak{-b+\sqrt{b^2-4ac}}}{2a}
\end{align}\\

The value of $x$ can then be substituted in \ref{eq:scalar3_quadilateral_ex} to find the coordinates of $\vec{C}$\\ 

\item
From the given information, 
The values are listed in \ref{table:table2_quadilateral_ex}\\
\begin{table}[ht!]
\centering
\input{./tables/quadilateral_ex/result}
\caption{Values of $\vec{C}$ and $\vec{D}$}
\label{table:table2_quadilateral_ex}	
\end{table} 

\item Draw Fig. \ref{fig:cyclic_quad2_quadilateral_ex}.

\begin{figure}[!ht]
\centering
\includegraphics[width=\columnwidth]{./figs/quadilateral_ex/cyclic_quad.eps}
\caption{Triangle generated using python}
\label{fig:cyclic_quad2_quadilateral_ex}
\end{figure} 

\solution The  following Python code generates Fig. \ref{fig:cyclic_quad2_quadilateral_ex}

\begin{lstlisting}
codes/quadilateral_ex/cyclic_quad.py
\end{lstlisting}

and the equivalent latex-tikz code generating Fig. \ref{fig:cyclic_quad2_quadilateral_ex} is 
\begin{lstlisting}
figs/quadilateral_ex/cyclic_quad_fig.tex
\end{lstlisting}
%
The above latex code can be compiled as a standalone document as
\begin{lstlisting}
figs/quadilateral_ex/cyclic_quad_final.tex
\end{lstlisting}
\end{enumerate}


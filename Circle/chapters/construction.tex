\renewcommand{\theequation}{\theenumi}
\begin{enumerate}[label=\thesection.\arabic*.,ref=\thesection.\theenumi]
\numberwithin{equation}{enumi}

\begin{figure}[!ht]
\centering
\resizebox{\columnwidth}{!}{\input{./figs/trapezium_altitude_fig}}
\caption{Trapezium by Latex-Tikz}
\label{fig:trapezium}	
\end{figure}

\item The figure obtained looks like Fig. \ref{fig:trapezium}.\\
 $AD = BC$, $AB \ ||\  DC$ . 

\item The design parameters used for construction See Table. \ref{table:table1}.
\begin{table}[ht!]
\centering
\input{./tables/design_params}
\caption{Trapezium $ABCD$}
\label{table:table1}	
\end{table} 

\item Find the coordinates of the various points in Fig
\begin{align}
\vec{A} &=\myvec{0\\0} \label{eq:constr_a}\\
\vec{B} &=\vec{A} + \myvec{a\\0}, \label{eq:constr_b}\\ 
\vec{AP_1} = -\vec{BP_2} &= \myvec{h\cot{\theta}\\0}\label{eq:constr_ap1}\\
\vec{P_1D} = \vec{P_2C} &= \myvec{0\\h}\label{eq:constr_bp2}
\end{align}
\begin{align}
\vec{C} &= \vec{B} - \vec{BP_2} + \vec{P_2C}\label{eq:constr_c}\\
\vec{D} &= \vec{A} + \vec{AP_1} + \vec{P_1D}\label{eq:constr_d}
\end{align}

\item
\solution From the given information, 
The values are listed in \ref{table:table2}\\
\begin{table}[ht!]
\centering
\input{./tables/results}
\caption{Values of $\vec{C}\  and\  \vec{D}$}
\label{table:table2}	
\end{table} 

\vskip75pt 
\item Draw Fig. \ref{fig:trapezium}.

\begin{figure}[!ht]
\centering
\includegraphics[width=\columnwidth]{./figs/trapezium_altitude.eps}
\caption{Trapezium generated using python}
\label{fig:trapezium2}
\end{figure} 

\solution The  following Python code generates Fig. \ref{fig:trapezium2}

\begin{lstlisting}
codes/quad.py
\end{lstlisting}

and the equivalent latex-tikz code generating Fig. \ref{fig:trapezium2} is 
\begin{lstlisting}
figs/trapezium_altitude_fig.tex
\end{lstlisting}
%
The above latex code can be compiled as a standalone document as
\begin{lstlisting}
figs/trapezium_final_altitude.tex
\end{lstlisting}
\end{enumerate}

